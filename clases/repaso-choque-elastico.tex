\documentclass[xetex,mathserif,serif,10pt]{beamer}
%\documentclass[xetex,mathserif,serif,10pt,handout]{beamer}

\usepackage{fontspec}
\usepackage{xunicode}
\usepackage{xltxtra}
\usepackage{graphicx}
\usepackage{bm}
\usepackage{comment}
\usepackage{multicol}
\usepackage[absolute,overlay,quiet]{textpos}
\usepackage{commath}
\usepackage{etaremune}
\usepackage{wasysym}

\defaultfontfeatures{Mapping=tex-text}
\setmainfont{Linux Libertine O}
\chardef\&="E050

\usepackage[margin=10pt,font=small,labelfont=bf,textfont=it,labelsep=colon,singlelinecheck=false,justification=justified]{caption}
\usepackage[spanish]{babel}
%\usepackage[final,expansion=true,protrusion=true,spacing=true,kerning=true]{microtype}

\mode<presentation> {
  % \hypersetup{pdfpagemode=FullScreen} %para poner en modo fullscreen de una
  % \usetheme{Boadilla} % Pretty neat, soft color.
  % \usecolortheme{orchid}
  \definecolor{chart00}{rgb}{0.00,0.00,0.00}
  \definecolor{chart01}{rgb}{0.00,0.27,0.53}
  \definecolor{chart02}{rgb}{1.00,0.26,0.05}
  \definecolor{chart03}{rgb}{1.00,0.83,0.13}
  \definecolor{chart04}{rgb}{0.34,0.62,0.11}
  \definecolor{chart05}{rgb}{0.49,0.00,0.13}
  \definecolor{chart06}{rgb}{0.51,1.15,1.00}
  \definecolor{chart07}{rgb}{0.19,0.25,0.02}
  \definecolor{chart08}{rgb}{0.68,0.81,0.00}
  \definecolor{chart09}{rgb}{0.29,0.12,0.44}
  \definecolor{chart10}{rgb}{1.00,0.58,0.05}
  \definecolor{chart11}{rgb}{0.77,0.00,0.04}
  \definecolor{chart12}{rgb}{0.00,0.52,0.82}
  \definecolor{chart13}{rgb}{1.00,1.00,1.00}
  \definecolor{beamer@blendedblue}{rgb}{0.,0.27,0.53}

  \setbeamercolor{normal text}{fg=black}
  \setbeamercolor{alerted text}{fg=chart11}
  \setbeamercolor{example text}{fg=chart08}
  %\setbeamercovered{transparent}%transparencia en las pausas
  %\beamertemplatetransparentcovereddynamicmedium
  %\setbeamertemplate{navigation symbols}{}
  \setbeamercolor{author}{fg=chart05}
}

\logo{\includegraphics[height=1.5cm]{logo-uis.png}}
\newcommand{\bblock}[1]{{\color{chart12}{#1}}}
\newcommand{\bc}[1]{
  \begin{center}
  #1
  \end{center}
}
\newcommand{\mb}[1]{\mathbf{#1}}

\newcommand{\be}[2]{
  \vspace{-0.5em}
  \begin{equation}\label{#2}
    #1
  \end{equation}
  \vspace{-1em}
}

\newcommand{\bet}[3]{
  \vspace{-0.5em}
  \begin{equation}\label{#2}
    \mathrm{#3:}\quad #1
  \end{equation}
  \vspace{-1em}
}

\def \unidad 	  {01}
\def \clase 	  {05}
\def \contenido {Colisiones}
\def \contone	  {colisionales}
\def \fecha 	  {20130917}
\def \dia	      {M}
\def \autor     {HA}
\def \file	{\unidad-\clase-\fecha\dia-\autor-\contone.pdf}

\title[\contone]{Introducción a la Física de Partículas\\\vspace*{1cm}\unidad-\clase\\\contenido}
\author[H. Asorey]{\Large{Hernán Asorey}}
\institute[hasorey@uis.edu.co]{
	Escuela de Física, Universidad Industrial de Santander\\
	Bucaramanga, Colombia\\
	\color{chart09}{\large{hasorey@uis.edu.co}}\\
	\color{chart05}{\large{\fecha\dia}}
}
\date[\fecha\dia]{\color{chart07}{\file}}

\begin{document}
%\tikzstyle{every picture}+=[remember picture]
%\everymath{\displaystyle}

\begin{comment}
\end{comment}
%%%%%%%%%%%%%%%%%%%%%%%
%\begin{frame}
%\titlepage
%\end{frame}

\logo{}

% \begin{textblock*}{22mm}(100mm,0.25\textheight)
% \begin{exampleblock}
% \tiny{Kotera et al, ARAA49:119(2011)53}
% \end{exampleblock}
% \end{textblock*}
% \begin{textblock*}{35mm}(90mm,0.00\textheight)
% 	\begin{alertblock}{Posibles fuentes}
% 		\begin{itemize}
% 			\item AGN
% 			\item Magnetars
% 			\item GRBs
% 		\end{itemize}
% 	\end{alertblock}
% \end{textblock*}

%%%%%%%%%%%%%%%%%%%%%%%%%%%%%%%%%%%%%%%%%%%
\section{Formulerío}
%%%%%%%%%%%%%%%%%%%%%%%%%%%%%%%%%%%%%%%%%%%

\begin{frame}
\frametitle{Choque elástico}
\begin{columns}
\column{0.6\textwidth}
\begin{alertblock}{Magnitudes conservadas}
\begin{itemize}
\item Energía total: $E_i=E_f$
\item Cantidad de movimiento: $\mathbf{p}_i = \mathbf{p}_f$
\end{itemize}
\end{alertblock}
\column{0.4\textwidth}
\begin{exampleblock}{Magnitudes constantes}
\begin{itemize}
\item Energía cinética: $E_{k,i}=E_{k,f}$
\end{itemize}
\end{exampleblock}
\end{columns}
Entonces, sean dos cuerpos de masas $m_1$ y $m_2$ moviéndose con velocidades iniciales $\mathbf{u}_1$ y $\mathbf{u}_2$. Luego del choque, sus velocidades finales serán $\mb{v}_1$ y $\mb{v}_2$:
\begin{itemize}
\item Conservación de la cantidad de movimiento: 
\be{
\mathbf{p}_i = \mathbf{p}_f \to m_1 \mb{u}_1 + m_2 \mb{u}_2 =  m_1 \mb{v}_1 + m_2 \mb{v}_2
}{EQImpulso}
\item Constancia de la energía cinética:
\be{
E_{k,i}=E_{k,f} \to \frac12 m_1 \mb{u}_1^2 + \frac12 m_2 \mb{u}_2^2 =  \frac12 m_1 \mb{v}_1^2 + \frac12 m_2 \mb{v}_2^2
}{EQCinetica}
y entonces
\be{
m_1 u_1^2 + m_2 u_2^2 =  m_1 v_1^2 + m_2 v_2^2
}{EQCin}
\end{itemize}
\begin{block}{Estado Final: 2 ecuaciones con 2 incognitas}
A partir de las condiciones iniciales, y sabiendo que es un choque elástico, ¿podemos determinar las condiciones finales ($\mb{v}_1$ y $\mb{v}_2$)?
\end{block}
\end{frame}

\begin{frame}
\frametitle{Álgebra}
\begin{enumerate}
\item Estamos en 1D, trabajamos con los módulos de las velocidades
\item Reordenamos (\ref{EQImpulso}), juntando las velocidades iniciales y finales de cada cuerpo:
\be{
    - m_1 (u_1 - v_1) = m_2 (u_2-v_2)
}{EQImp1}
\item y lo mismo para la energía cinética (\ref{EQCin}):
\be{
m_2 (u_2^2 - v_2^2) = - m_1 (u_1^2 - v_1^2) \nonumber
}{EQCin0}
\item usando diferencia de cuadrados, $a^2-b^2=(a-b)(a+b)$,
\be{
m_2 (u_2-v_2)(u_2+v_2) = - m_1 (u_1 - v_1) (u_1+v_1)
}{EQCin1}
\item mirando fijamente y comparando (\ref{EQImp1}) con (\ref{EQCin1}), vemos que:
\be{u_2+b_2 = u_1 + v_1 \to (u_2 - u_1) = - (v_2 - v_1) \to \Delta u = - \Delta v
}{EQCin2}
\item con lo cual, podemos despejar, por ejemplo, $v_2$:
\be{v_2=u_1+v_1-u2}{EQVels}
\end{enumerate}
\end{frame}

\begin{frame}
\frametitle{Más álgebra, ya casi}
\begin{enumerate}
\setcounter{enumi}{6}      %% Offset en numero de problema
\item Podemos utilizar (\ref{EQVels}), para poner todo en función de $v_1$, y despejar $v_1$. Partimos de (\ref{EQImp1}):
\be{
    m_2(u_2 - u_1 + u_2 - v_1) = - m_1 (u_1 - v_1)
}{EQVel0}
\item y tratamos de juntar las velocidades $v_1$:
\be{
    m_2(2 u_2 - u_1) - m_2 v_1 = - m_1 u_1 + m_1 v_1
}{EQVel1}
\item insistimos,
\begin{eqnarray*}
    m_2 (2 u_2 - u_1) + m_1 u_1 &=& (m_1 + m_2) v_1 \\
    2 m_2 u_2 - m_2 u_1 + m_1 u_1 &=& (m_1 + m_2) v_1 \\
    2 m_2 u_2 - (m_1 - m_2) u_1 &=& (m_1 + m_2) v_1 \\
\end{eqnarray*}
\vspace{-1em}
\item y finalmente,
\be{
    v_1 = \frac{m_1 - m_2}{m_1 + m_2} u_1 + \frac{2 m_2}{m_1+m_2} u_2
}{EQV1}

\item Cambiando los índices $1 \leftrightarrow 2$, obtenemos $v_2$:
\be{
    v_2 = \frac{2 m_1}{m_1+m_2} u_1 - \frac{m_1-m_2}{m_1+m_2} u_2
}{EQV2}
\end{enumerate}
\end{frame}

\begin{frame}
\frametitle{Casos límites}
\begin{itemize}
\item \bblock{autos chocadores}, $m_1=m_2$: \alert{¡Las velocidades se intercambian!}
\begin{eqnarray*}
    v_1 = \frac{m_1 - m_2}{m_1 + m_2} u_1 + \frac{2 m_2}{m_1+m_2} u_2 &\to& v_1 = u_2 \\
    v_2 = \frac{2 m_1}{m_1+m_2} u_1 - \frac{m_1-m_2}{m_1+m_2} u_2 &\to& v_2 = u_1
\end{eqnarray*}
\vspace{-1em}

\item \bblock{Billar}, $m_1=m_2, u_2=0$: \alert{¡La primera bola se queda quieta!}
\begin{eqnarray*}
    v_1 = \frac{m_1 - m_2}{m_1 + m_2} u_1 + \frac{2 m_2}{m_1+m_2} u_2 &\to& v_1 = 0 \\
    v_2 = \frac{2 m_1}{m_1+m_2} u_1 - \frac{m_1-m_2}{m_1+m_2} u_2 &\to& v_2 = u_1
\end{eqnarray*}
\vspace{-1em}

\item \bblock{Camión vs taxi, elástico}, $m_1 \gg m_2$: \alert{Pobre taxista...}
\begin{eqnarray*}
    v_1 = \frac{m_1 - m_2}{m_1 + m_2} u_1 + \frac{2 m_2}{m_1+m_2} u_2 &\to&  v_1 \approx u_1 \\
    v_2 = \frac{2 m_1}{m_1+m_2} u_1 - \frac{m_1-m_2}{m_1+m_2} u_2 &\to& v_2 \approx 2 u_1
\end{eqnarray*}
\vspace{-1em}
\end{itemize}
\end{frame}

\begin{frame}
\frametitle{Casos límites}
\begin{itemize}
\item \bblock{Choque contra una pared}, $u_2=0, m_2\to\infty$: \alert{¡Rebote!}
\\
{\tiny{(el viejo truco, saco $m_2$ como factor común, y hago tender al límite)}}
\begin{eqnarray*}
    v_1 = \frac{m_1 - m_2}{m_1 + m_2} u_1 + \frac{2 m_2}{m_1+m_2} u_2 &\to& v_1 \simeq - u_1 \\
    v_2 = \frac{2 m_1}{m_1+m_2} u_1 - \frac{m_1-m_2}{m_1+m_2} u_2 &\to& v_2 \simeq 0
\end{eqnarray*}
\vspace{-1em}
\item Vectorialmente, cambia sólo la coordenada donde está la pared.
\item Imaginemos una pelota de masa $m$ con velocidad $\mb{u}=(u_x,u_y,u_z)$, que choca una pared en $x=1$. Al llegar a $x=1$, entonces
\begin{eqnarray*}
    v_x &=& - u_x \\
    v_y &=&  u_y \\
    v_z &=&  u_z 
\end{eqnarray*}
\begin{center}\alert{Pensar una pelota chocando contra una pared}\end{center}
\item La velocidad final es entonces $\mb{v}=(-u_x,u_y,u_z)$.
\end{itemize}

\end{frame}



\end{document}


\end{document}
