\documentclass[xetex,mathserif,serif,10pt]{beamer}
%\documentclass[xetex,mathserif,serif,10pt,handout]{beamer}

\usepackage{fontspec}
\usepackage{xunicode}
\usepackage{xltxtra}
\usepackage{graphicx}
\usepackage{bm}
\usepackage{comment}
\usepackage{multicol}
\usepackage[absolute,overlay,quiet]{textpos}
\usepackage{commath}
\usepackage{etaremune}
\usepackage{wasysym}

\defaultfontfeatures{Mapping=tex-text}
\setmainfont{Linux Libertine O}
\chardef\&="E050

\usepackage[margin=10pt,font=small,labelfont=bf,textfont=it,labelsep=colon,singlelinecheck=false,justification=justified]{caption}
\usepackage[spanish]{babel}
\usepackage[final,expansion=true,protrusion=true,spacing=true,kerning=true]{microtype}

\mode<presentation> {
  %\hypersetup{pdfpagemode=FullScreen} %para poner en modo fullscreen de una
  \usetheme{Boadilla} % Pretty neat, soft color.
  \usecolortheme{orchid}
  \definecolor{chart00}{rgb}{0.00,0.00,0.00}
  \definecolor{chart01}{rgb}{0.00,0.27,0.53}
  \definecolor{chart02}{rgb}{1.00,0.26,0.05}
  \definecolor{chart03}{rgb}{1.00,0.83,0.13}
  \definecolor{chart04}{rgb}{0.34,0.62,0.11}
  \definecolor{chart05}{rgb}{0.49,0.00,0.13}
  \definecolor{chart06}{rgb}{0.51,1.15,1.00}
  \definecolor{chart07}{rgb}{0.19,0.25,0.02}
  \definecolor{chart08}{rgb}{0.68,0.81,0.00}
  \definecolor{chart09}{rgb}{0.29,0.12,0.44}
  \definecolor{chart10}{rgb}{1.00,0.58,0.05}
  \definecolor{chart11}{rgb}{0.77,0.00,0.04}
  \definecolor{chart12}{rgb}{0.00,0.52,0.82}
  \definecolor{chart13}{rgb}{1.00,1.00,1.00}
  \definecolor{beamer@blendedblue}{rgb}{0.,0.27,0.53}
  \setbeamercolor{normal text}{fg=black}
  \setbeamercolor{alerted text}{fg=chart11}
  \setbeamercolor{example text}{fg=chart08}
  %\setbeamercovered{transparent}%transparencia en las pausas
  %\beamertemplatetransparentcovereddynamicmedium
  %\setbeamertemplate{navigation symbols}{}
  \setbeamercolor{author}{fg=chart05}
}

%\hspace{10em}
%\logo{\includegraphics[height=1.0cm]{garland_iteda_logo.png}}
%


\def \unidad 	{01}
\def \clase 	{01}
\def \contenido {Introducción general}
\def \contone	{introduccion}
\def \fecha 	{20130903}
\def \dia	{M}
\def \autor     {HA}
\def \file	{\unidad-\clase-\fecha\dia-\autor-\contone.pdf}

\title[\contone]{Introducción a la Física de Partículas\\\vspace*{1cm}\unidad-\clase\\\contenido}
\author[H. Asorey]{\Large{Hernán Asorey}}
\institute[hasorey@uis.edu.co]{
	Escuela de Física, Universidad Industrial de Santander\\
	Bucaramanga, Colombia\\
	\color{chart09}{\large{hasorey@uis.edu.co}}\\
	\color{chart05}{\large{\fecha\dia}}
}
\date[\fecha\dia]{\color{chart07}{\file}}

\begin{document}
%\tikzstyle{every picture}+=[remember picture]
%\everymath{\displaystyle}

\begin{comment}
\end{comment}
%%%%%%%%%%%%%%%%%%%%%%%
\begin{frame}
\titlepage
\end{frame}

%%%%%%%%%%%%%%%%%%%%%%%%%%%%%%%%%%%%%%%%%%%
\section{Rayos cósmicos}
%%%%%%%%%%%%%%%%%%%%%%%%%%%%%%%%%%%%%%%%%%%




\begin{frame}
\Huge{\alert{Los Rayos Cósmicos}}
\begin{columns}
\column{0.50\textwidth}
%{\centering \includegraphics[width=0.55\columnwidth]{./figs/rc/hess_c.jpg}}
\column{0.50\textwidth}
\vspace*{1cm}
\begin{flushright}
\small{
{\it{The results of my observations are best explained by the assumption that a\\
radiation of very great penetrating power enters our atmosphere from above.}}\\
- Victor Hess\\
Physikalische Zeitschrift (1912)
}
\end{flushright}
\end{columns}
\end{frame}

\begin{frame}
	\frametitle{Producción}
	\framesubtitle{Aceleración y Decaimiento}
	\begin{exampleblock}{Aceleración: Fermi + difusión}
		\begin{columns}
			\column{0.60\textwidth}
			\begin{itemize}
				\item Segundo orden: ($\beta_c \to 0$): $\overline{\Delta E}/E \propto \beta_c^2$
				\item Primer orden: (SNR, $\beta_s \sim 0.03$): $\overline{\Delta E}/E \simeq \beta_s$
				\item Espectro tipo ley de potencia:
					\begin{equation*}
					\alpha = \frac{\ln (1-p_\mathrm{esc})}{\ln \left (1 + \overline{\Delta E}/E \right)} -1 \to \alpha = -2
				\end{equation*}
		\end{itemize}
		\column{0.40\textwidth}
%		\includegraphics[width=0.90\columnwidth]{figs/rc/hillas_kotera2011_c.png}
		% \begin{textblock*}{22mm}(100mm,0.25\textheight)
		% \begin{exampleblock}
		% \tiny{Kotera et al, ARAA49:119(2011)53}
		% \end{exampleblock}
		% \end{textblock*}
	\end{columns}
\end{exampleblock}

\begin{textblock*}{35mm}(90mm,0.00\textheight)
	\begin{alertblock}{Posibles fuentes}
		\begin{itemize}
			\item AGN
			\item Magnetars
			\item GRBs
		\end{itemize}
	\end{alertblock}
\end{textblock*}
\begin{block}{Decaimiento}<2->
	\begin{itemize}
		\item Decaimiento de partículas supermasivas $X$, $m_X \gtrsim 10^{21}$\,eV
		\item Todos los modelos predicen $n_\gamma \simeq n_\nu \gg n_{\mathrm{nucleones}}$
		\item Flujos predichos inconsistentes con las observaciones de composición a UHE
	\end{itemize}
\end{block}

\end{frame}


\end{document} 
